% Options for packages loaded elsewhere
\PassOptionsToPackage{unicode}{hyperref}
\PassOptionsToPackage{hyphens}{url}
%
\documentclass[
]{article}
\usepackage{amsmath,amssymb}
\usepackage{lmodern}
\usepackage{ifxetex,ifluatex}
\ifnum 0\ifxetex 1\fi\ifluatex 1\fi=0 % if pdftex
  \usepackage[T1]{fontenc}
  \usepackage[utf8]{inputenc}
  \usepackage{textcomp} % provide euro and other symbols
\else % if luatex or xetex
  \usepackage{unicode-math}
  \defaultfontfeatures{Scale=MatchLowercase}
  \defaultfontfeatures[\rmfamily]{Ligatures=TeX,Scale=1}
\fi
% Use upquote if available, for straight quotes in verbatim environments
\IfFileExists{upquote.sty}{\usepackage{upquote}}{}
\IfFileExists{microtype.sty}{% use microtype if available
  \usepackage[]{microtype}
  \UseMicrotypeSet[protrusion]{basicmath} % disable protrusion for tt fonts
}{}
\makeatletter
\@ifundefined{KOMAClassName}{% if non-KOMA class
  \IfFileExists{parskip.sty}{%
    \usepackage{parskip}
  }{% else
    \setlength{\parindent}{0pt}
    \setlength{\parskip}{6pt plus 2pt minus 1pt}}
}{% if KOMA class
  \KOMAoptions{parskip=half}}
\makeatother
\usepackage{xcolor}
\IfFileExists{xurl.sty}{\usepackage{xurl}}{} % add URL line breaks if available
\IfFileExists{bookmark.sty}{\usepackage{bookmark}}{\usepackage{hyperref}}
\hypersetup{
  pdftitle={CSC8631 Report},
  pdfauthor={Yimiao Wang},
  hidelinks,
  pdfcreator={LaTeX via pandoc}}
\urlstyle{same} % disable monospaced font for URLs
\usepackage[margin=1in]{geometry}
\usepackage{color}
\usepackage{fancyvrb}
\newcommand{\VerbBar}{|}
\newcommand{\VERB}{\Verb[commandchars=\\\{\}]}
\DefineVerbatimEnvironment{Highlighting}{Verbatim}{commandchars=\\\{\}}
% Add ',fontsize=\small' for more characters per line
\usepackage{framed}
\definecolor{shadecolor}{RGB}{248,248,248}
\newenvironment{Shaded}{\begin{snugshade}}{\end{snugshade}}
\newcommand{\AlertTok}[1]{\textcolor[rgb]{0.94,0.16,0.16}{#1}}
\newcommand{\AnnotationTok}[1]{\textcolor[rgb]{0.56,0.35,0.01}{\textbf{\textit{#1}}}}
\newcommand{\AttributeTok}[1]{\textcolor[rgb]{0.77,0.63,0.00}{#1}}
\newcommand{\BaseNTok}[1]{\textcolor[rgb]{0.00,0.00,0.81}{#1}}
\newcommand{\BuiltInTok}[1]{#1}
\newcommand{\CharTok}[1]{\textcolor[rgb]{0.31,0.60,0.02}{#1}}
\newcommand{\CommentTok}[1]{\textcolor[rgb]{0.56,0.35,0.01}{\textit{#1}}}
\newcommand{\CommentVarTok}[1]{\textcolor[rgb]{0.56,0.35,0.01}{\textbf{\textit{#1}}}}
\newcommand{\ConstantTok}[1]{\textcolor[rgb]{0.00,0.00,0.00}{#1}}
\newcommand{\ControlFlowTok}[1]{\textcolor[rgb]{0.13,0.29,0.53}{\textbf{#1}}}
\newcommand{\DataTypeTok}[1]{\textcolor[rgb]{0.13,0.29,0.53}{#1}}
\newcommand{\DecValTok}[1]{\textcolor[rgb]{0.00,0.00,0.81}{#1}}
\newcommand{\DocumentationTok}[1]{\textcolor[rgb]{0.56,0.35,0.01}{\textbf{\textit{#1}}}}
\newcommand{\ErrorTok}[1]{\textcolor[rgb]{0.64,0.00,0.00}{\textbf{#1}}}
\newcommand{\ExtensionTok}[1]{#1}
\newcommand{\FloatTok}[1]{\textcolor[rgb]{0.00,0.00,0.81}{#1}}
\newcommand{\FunctionTok}[1]{\textcolor[rgb]{0.00,0.00,0.00}{#1}}
\newcommand{\ImportTok}[1]{#1}
\newcommand{\InformationTok}[1]{\textcolor[rgb]{0.56,0.35,0.01}{\textbf{\textit{#1}}}}
\newcommand{\KeywordTok}[1]{\textcolor[rgb]{0.13,0.29,0.53}{\textbf{#1}}}
\newcommand{\NormalTok}[1]{#1}
\newcommand{\OperatorTok}[1]{\textcolor[rgb]{0.81,0.36,0.00}{\textbf{#1}}}
\newcommand{\OtherTok}[1]{\textcolor[rgb]{0.56,0.35,0.01}{#1}}
\newcommand{\PreprocessorTok}[1]{\textcolor[rgb]{0.56,0.35,0.01}{\textit{#1}}}
\newcommand{\RegionMarkerTok}[1]{#1}
\newcommand{\SpecialCharTok}[1]{\textcolor[rgb]{0.00,0.00,0.00}{#1}}
\newcommand{\SpecialStringTok}[1]{\textcolor[rgb]{0.31,0.60,0.02}{#1}}
\newcommand{\StringTok}[1]{\textcolor[rgb]{0.31,0.60,0.02}{#1}}
\newcommand{\VariableTok}[1]{\textcolor[rgb]{0.00,0.00,0.00}{#1}}
\newcommand{\VerbatimStringTok}[1]{\textcolor[rgb]{0.31,0.60,0.02}{#1}}
\newcommand{\WarningTok}[1]{\textcolor[rgb]{0.56,0.35,0.01}{\textbf{\textit{#1}}}}
\usepackage{longtable,booktabs,array}
\usepackage{calc} % for calculating minipage widths
% Correct order of tables after \paragraph or \subparagraph
\usepackage{etoolbox}
\makeatletter
\patchcmd\longtable{\par}{\if@noskipsec\mbox{}\fi\par}{}{}
\makeatother
% Allow footnotes in longtable head/foot
\IfFileExists{footnotehyper.sty}{\usepackage{footnotehyper}}{\usepackage{footnote}}
\makesavenoteenv{longtable}
\usepackage{graphicx}
\makeatletter
\def\maxwidth{\ifdim\Gin@nat@width>\linewidth\linewidth\else\Gin@nat@width\fi}
\def\maxheight{\ifdim\Gin@nat@height>\textheight\textheight\else\Gin@nat@height\fi}
\makeatother
% Scale images if necessary, so that they will not overflow the page
% margins by default, and it is still possible to overwrite the defaults
% using explicit options in \includegraphics[width, height, ...]{}
\setkeys{Gin}{width=\maxwidth,height=\maxheight,keepaspectratio}
% Set default figure placement to htbp
\makeatletter
\def\fps@figure{htbp}
\makeatother
\setlength{\emergencystretch}{3em} % prevent overfull lines
\providecommand{\tightlist}{%
  \setlength{\itemsep}{0pt}\setlength{\parskip}{0pt}}
\setcounter{secnumdepth}{-\maxdimen} % remove section numbering
\ifluatex
  \usepackage{selnolig}  % disable illegal ligatures
\fi

\title{CSC8631 Report}
\author{Yimiao Wang}
\date{11/25/2021}

\begin{document}
\maketitle

\hypertarget{data-merge-and-cleaning}{%
\subsection{Data Merge and Cleaning}\label{data-merge-and-cleaning}}

Use the rbind() function to merge the data sets into a table according
to different names, and use the duplicated function to delete the
duplicate values in the data, and use the na.omit() function to delete
rows with NA values in the data. For archetype Survey responses and
enrolment data delete rows with duplicate values of learner id.

\hypertarget{exploratory-data-analysis}{%
\subsection{Exploratory data analysis}\label{exploratory-data-analysis}}

\hypertarget{enrolments}{%
\subsubsection{Enrolments}\label{enrolments}}

First use the 7 tables of enrolments for data visualization
analysis,load the dplyr package,delete the rows with unknown and ``-''
in the data set,and extract the
gender,age\_range,country,highest\_education,employment\_status,and
employee\_area in the table,respectively use data visualization
processing and analyze the size of variables between different students.

\begin{verbatim}
## 
## 载入程辑包:'dplyr'
\end{verbatim}

\begin{verbatim}
## The following objects are masked from 'package:stats':
## 
##     filter, lag
\end{verbatim}

\begin{verbatim}
## The following objects are masked from 'package:base':
## 
##     intersect, setdiff, setequal, union
\end{verbatim}

From the gender variable, it can be seen that in all enrolments data,
there are more males than females. Draw the composition of different age
groups. It can be seen from the figure that most students are between 26
to 35 years old, and With the increase of age, the number of students
decreases to a certain extent. It is worth noting that the number of
people younger than 18 years old is at least lower than that of middle
school. Therefore, it can be clearly seen from the educational
distribution chart that the highest degree of most students is a
bachelor degree, and the second most number is a master degree. Among
all scholars, the lowest degree is lower than that of middle school, and
the number of college degree is the least.

\begin{Shaded}
\begin{Highlighting}[]
\FunctionTok{library}\NormalTok{(ggplot2)}
\FunctionTok{library}\NormalTok{(fmsb)}
\CommentTok{\#Visualize gender,agerange,education,detected country,employment}
\FunctionTok{ggplot}\NormalTok{(gender,}\FunctionTok{aes}\NormalTok{(}\AttributeTok{x=}\NormalTok{Var1,}\AttributeTok{y=}\NormalTok{Freq,}\AttributeTok{fill=}\NormalTok{gender[,}\DecValTok{1}\NormalTok{],}\AttributeTok{group=}\FunctionTok{factor}\NormalTok{(}\DecValTok{1}\NormalTok{)))}\SpecialCharTok{+}\FunctionTok{geom\_bar}\NormalTok{(}\AttributeTok{stat=}\StringTok{"identity"}\NormalTok{)}
\end{Highlighting}
\end{Shaded}

\includegraphics{Report_files/figure-latex/unnamed-chunk-3-1.pdf}

\begin{Shaded}
\begin{Highlighting}[]
\FunctionTok{ggplot}\NormalTok{(age\_range,}\FunctionTok{aes}\NormalTok{(}\AttributeTok{x=}\NormalTok{Var1,}\AttributeTok{y=}\NormalTok{Freq,}\AttributeTok{fill=}\NormalTok{age\_range[,}\DecValTok{1}\NormalTok{],}\AttributeTok{group=}\FunctionTok{factor}\NormalTok{(}\DecValTok{1}\NormalTok{)))}\SpecialCharTok{+}\FunctionTok{geom\_point}\NormalTok{(}\AttributeTok{stat=}\StringTok{"identity"}\NormalTok{)}
\end{Highlighting}
\end{Shaded}

\includegraphics{Report_files/figure-latex/unnamed-chunk-3-2.pdf}

\begin{Shaded}
\begin{Highlighting}[]
\FunctionTok{ggplot}\NormalTok{(highest\_education,}\FunctionTok{aes}\NormalTok{(}\AttributeTok{x=}\NormalTok{Var1,}\AttributeTok{y=}\NormalTok{Freq,}\AttributeTok{fill=}\NormalTok{highest\_education[,}\DecValTok{1}\NormalTok{],}\AttributeTok{group=}\FunctionTok{factor}\NormalTok{(}\DecValTok{1}\NormalTok{)))}\SpecialCharTok{+}\FunctionTok{geom\_bar}\NormalTok{(}\AttributeTok{stat=}\StringTok{"identity"}\NormalTok{)}
\end{Highlighting}
\end{Shaded}

\includegraphics{Report_files/figure-latex/unnamed-chunk-3-3.pdf}

Since the scholars come from different countries and regions, a
histogram is made to check the distribution of the countries. From the
figure, it can be seen that most of the scholars are from the United
Kingdom, and the number of other countries is significantly smaller. The
number of people from the three countries of MX, PK, and ES is the
least.

\includegraphics{Report_files/figure-latex/unnamed-chunk-4-1.pdf}
\includegraphics{Report_files/figure-latex/unnamed-chunk-4-2.pdf}
\includegraphics{Report_files/figure-latex/unnamed-chunk-4-3.pdf}

There are many professional workers among all learners, so visualize the
employment area and analyze the work fields of different learners. From
the histogram, it can be seen that the number of people engaged in it
and information services accounts for most of the proportion, while
those engaged in teaching and education the number is second, and the
occupation with the least number is recruitment and pr.analyze the
employment status of different scholars, load the fmsb program package,
and make a radar chart. From the radar chart, it can be intuitively seen
that the largest employment status is full-time employment, of which the
second number is retired retirees,the number of unemployed is the least.

\hypertarget{step-activity}{%
\subsubsection{Step activity}\label{step-activity}}

In order to be able to know the time required for students to do a
question, use the data in the step activity to calculate the time
difference,first delete the data with null values in the
last\_completed\_at data set, take question 1.1 as the main analysis
object,first filter out all answers to the 1.1 question People, extract
the 5th and 6th columns, namely the time of the first visit and the
completion time of the student, use the difftime() function to calculate
the time difference, where as.POSIXct is used to specify the format of
the year, month, day, time and time zone, and the apply() function is
used to find the largest time difference With the minimum time
difference, the maximum value is 56516879 seconds in line 58, which
takes 15 hours to complete, and the minimum value is 1 second. Most of
the time difference is between {[}500,1000{]}, indicating that the
problem of 1.1 there are some more difficult questions.Then extract a
question in 1.1, take 1.14 as an example, and also calculate the time
difference. The maximum value is 41476865 seconds, and the minimum value
is also 1 second.

\begin{longtable}[]{@{}cccc@{}}
\toprule
& interval & max & min \\
\midrule
\endhead
1.1 & {[}1000,4000{]} & 15h & 1s \\
1.14 & {[}500,1000{]} & 14h & 1s \\
:---: & :---: & :---: & :---: \\
\bottomrule
\end{longtable}

\hypertarget{question-response}{%
\subsubsection{Question response}\label{question-response}}

In order to get a clearer understanding of students
answering,integrating the 7 tables of question response, delete all
empty values in the 8th column, and delete rows with NA in the table to
extract the number of people who answered 1.7.1-1.7.6. Since there will
be duplicate students doing 1.7.1 and the answer is TRUE in the data
set, the duplicate value is deleted, and the number of people who
answered the question correctly is 3143. Similarly, count the number of
people whose answer is wrong, and calculate Calculate the total number
of people who answered the question and calculate the total error rate.
Among the number of people who answered incorrectly, there are some
people who repeatedly answered incorrectly. Use the total number of
wrong answers minus the number of wrong answers to get the number of
repeated mistakes, and repeat the wrong answers. Divide the number of
people by the total number of incorrect answers to get the repeat error
rate. Questions with a high repeat error rate indicate that they are
more difficult and need to be explained.

\includegraphics{Report_files/figure-latex/unnamed-chunk-5-1.pdf}
\includegraphics{Report_files/figure-latex/unnamed-chunk-5-2.pdf}
\includegraphics{Report_files/figure-latex/unnamed-chunk-5-3.pdf}
Therefore, in order to see the total error rate, repeated error rate of
each question more intuitively, draw a table, a pie chart, and a
histogram. It can be seen that 1.7.5 and 1.7.2 have the lowest repeat
error rate, and the number of correct answers is also High, and 1.7.3 is
the question with the highest repeat error rate, indicating that it is
necessary to focus on the question.

\hypertarget{video-stats}{%
\subsubsection{Video stats}\label{video-stats}}

In order to be able to visually see the positive status of students
learning, the video status data set is merged. First, the data is
sorted, and all columns with 0 are deleted to obtain a data set with 65
rows and 23 columns. Use ggplot to make a density map from the video. It
can be seen from the density graph of the total playback volume that the
video playback volume of each section has the highest density at 500,
indicating that the average number of views is about 500. From the
density graph of the number of video downloads, it can be seen that the
number of video downloads in each section is About 50 times.

\includegraphics{Report_files/figure-latex/unnamed-chunk-6-1.pdf}
\includegraphics{Report_files/figure-latex/unnamed-chunk-6-2.pdf}

In order to further study the relationship between variables, take rows
1 to 13 that is the data in the first table, load the reshape2 program
package to refit the data set, take 1 to 4 columns, and make a scatter
plot, from the scatter plot It can be seen that there is a linear
relationship between the total number of videos played and the number of
downloads. From the scattered point distribution, it can be seen that
the most viewed and downloaded chapters of the video are 1.1, followed
by 1.5, and the third chapter is generally very small, indicating that
students enthusiasm for learning will be reduced later.

\includegraphics{Report_files/figure-latex/unnamed-chunk-7-1.pdf}

Making an area chart for whether learner can understand the video
content without adding subtitles. It can be seen from the figure that
section 1.1 and section 1.5 have the most views, indicating that these
two video contents are very Easy to understand. As for the requirements
for high-definition video, it can be seen in the area chart that section
2.11 uses high-definition viewing the most, indicating that the 2.11
video definition needs to be improved.

\includegraphics{Report_files/figure-latex/unnamed-chunk-8-1.pdf}
\includegraphics{Report_files/figure-latex/unnamed-chunk-8-2.pdf}

People who watch the video are in Europe, with a median of about 65.5\%,
indicating that the average view of people in Europe is 65.5\%. The
proportion of viewing in Asia and Oceania has decreased significantly,
with the medians being 9.75\% and 3.25\% respectively.

\includegraphics{Report_files/figure-latex/unnamed-chunk-9-1.pdf}
\includegraphics{Report_files/figure-latex/unnamed-chunk-9-2.pdf}
\includegraphics{Report_files/figure-latex/unnamed-chunk-9-3.pdf}

\hypertarget{model-establishment}{%
\subsection{Model establishment}\label{model-establishment}}

\hypertarget{video-stats-least-squares-regression-model}{%
\subsubsection{video stats least squares regression
model}\label{video-stats-least-squares-regression-model}}

To model the video stats, first do the principal component analysis on
the data, delete the data in the second column and columns 15 to 23,
select the data in rows 14 to 26 for principal component analysis, first
calculate the correlation between the variables, It can be seen that
there is a strong correlation between the first three variables. Use the
var function to obtain the sample covariance matrix and extract the
diagonal elements. The total coefficient of variation is calculated to
be 181379.7.

\begin{verbatim}
## 
## 载入程辑包:'MASS'
\end{verbatim}

\begin{verbatim}
## The following object is masked from 'package:dplyr':
## 
##     select
\end{verbatim}

\begin{verbatim}
##                           total_views total_downloads total_caption_views
## total_views                1.00000000      0.98642400          0.90324840
## total_downloads            0.98642400      1.00000000          0.86109753
## total_caption_views        0.90324840      0.86109753          1.00000000
## total_transcript_views     0.94292177      0.94308561          0.77333149
## viewed_hd                 -0.02184285     -0.01030225          0.08034915
## viewed_five_percent        0.72336421      0.68836373          0.59368574
## viewed_ten_percent         0.67096273      0.61282545          0.59109243
## viewed_twentyfive_percent  0.28672817      0.19681943          0.38300383
##                           total_transcript_views   viewed_hd
## total_views                           0.94292177 -0.02184285
## total_downloads                       0.94308561 -0.01030225
## total_caption_views                   0.77333149  0.08034915
## total_transcript_views                1.00000000 -0.02963948
## viewed_hd                            -0.02963948  1.00000000
## viewed_five_percent                   0.58473554 -0.34077028
## viewed_ten_percent                    0.52648630 -0.50038967
## viewed_twentyfive_percent             0.08773656 -0.59430671
##                           viewed_five_percent viewed_ten_percent
## total_views                         0.7233642          0.6709627
## total_downloads                     0.6883637          0.6128255
## total_caption_views                 0.5936857          0.5910924
## total_transcript_views              0.5847355          0.5264863
## viewed_hd                          -0.3407703         -0.5003897
## viewed_five_percent                 1.0000000          0.9316016
## viewed_ten_percent                  0.9316016          1.0000000
## viewed_twentyfive_percent           0.6648653          0.8424012
##                           viewed_twentyfive_percent
## total_views                              0.28672817
## total_downloads                          0.19681943
## total_caption_views                      0.38300383
## total_transcript_views                   0.08773656
## viewed_hd                               -0.59430671
## viewed_five_percent                      0.66486535
## viewed_ten_percent                       0.84240118
## viewed_twentyfive_percent                1.00000000
\end{verbatim}

\begin{verbatim}
##               total_views           total_downloads       total_caption_views 
##              1.605838e+05              1.829077e+03              8.907692e+01 
##    total_transcript_views                 viewed_hd       viewed_five_percent 
##              2.776692e+03              1.607273e+04              7.525924e+00 
##        viewed_ten_percent viewed_twentyfive_percent 
##              9.238994e+00              1.150693e+01
\end{verbatim}

Use the prcomp() function to find the principal components and view the
summary. The standard deviation of each principal component exceeds 100,
and the cumulative variance contribution rate reaches 90\%. It can be
seen from the gravel graph that the variance of the first two principal
components accounts for most of the total variance, so the first two
principal components are selected. The ingredients are appropriate.

\begin{verbatim}
## Importance of components:
##                             PC1       PC2      PC3     PC4     PC5     PC6
## Standard deviation     406.1134 126.77751 17.79552 6.98204 3.21674 1.59953
## Proportion of Variance   0.9093   0.08861  0.00175 0.00027 0.00006 0.00001
## Cumulative Proportion    0.9093   0.99791  0.99966 0.99993 0.99998 1.00000
##                           PC7    PC8
## Standard deviation     0.7727 0.2566
## Proportion of Variance 0.0000 0.0000
## Cumulative Proportion  1.0000 1.0000
\end{verbatim}

\includegraphics{Report_files/figure-latex/unnamed-chunk-11-1.pdf}

The next step is to fit the least squares regression model. When doing
correlation analysis, it is found that there is no correlation between
the following variables and the total amount of video playback.
Therefore, columns 3 to 6 of the data set are selected as independent
variable, total viewed as the dependent variable, check the model
summary and found that the first two variables are significant, that is,
the total number of videos played has a significant relationship with
the number of downloads and the number of views, draw the residual
diagram, and the distribution of the residual diagram on the left It is
rather scattered, and most of them are randomly distributed around 0.

\begin{verbatim}
## 
## Call:
## lm(formula = y ~ ., data = video_data)
## 
## Residuals:
##     Min      1Q  Median      3Q     Max 
## -63.853 -24.106   1.286  14.605  80.508 
## 
## Coefficients:
##                        Estimate Std. Error t value Pr(>|t|)    
## (Intercept)              792.85      14.14  56.064 1.14e-11 ***
## total_downloads          235.61      56.82   4.147  0.00322 ** 
## total_caption_views       99.36      30.16   3.294  0.01095 *  
## total_transcript_views    78.46      45.48   1.725  0.12279    
## viewed_hd                -11.98      14.96  -0.801  0.44610    
## ---
## Signif. codes:  0 '***' 0.001 '**' 0.01 '*' 0.05 '.' 0.1 ' ' 1
## 
## Residual standard error: 50.99 on 8 degrees of freedom
## Multiple R-squared:  0.9892, Adjusted R-squared:  0.9838 
## F-statistic: 183.3 on 4 and 8 DF,  p-value: 6.728e-08
\end{verbatim}

\includegraphics{Report_files/figure-latex/unnamed-chunk-12-1.pdf}

Therefore, it can be considered that the total views volume of the video
has a linear relationship with the number of downloads and the viewing
time. The right panel is a quantile graph to evaluate the residuals. The
normality of, most of the points are close to the unit diagonal,
indicating that the residual of the linear regression model is normal.

\hypertarget{question-response-logistic-model}{%
\subsubsection{Question response logistic
model}\label{question-response-logistic-model}}

Perform model fitting on the question response data set. Since the data
set type of the correct column is character, it needs to be converted to
an integer type, and replace false with the number 2 and true with the
number 1. Delete the data in the eighth column to Take the number of
people who answered question 1.8.1 as an example, set a random seed,
divide the data set into training set and test set according to 8:2, and
delete the comma in the response column, use 1 for false, 0 for true,
and correct as the response Variables and response columns are used as
predictors to test the correlation between variables, fit a logistic
model, and make predictions. The final test error is 0, indicating that
the model has a better predictive effect.It can be seen from the figure
that when the student's answer is 123, it can be counted as correct, and
all other answers are counted as wrong.

\begin{verbatim}
## [1] 2
\end{verbatim}

\begin{verbatim}
## [1] 1
\end{verbatim}

\begin{verbatim}
## 
##    0    1 
## 1104  824
\end{verbatim}

\begin{verbatim}
## 
## Call:
## glm(formula = correct ~ t, data = quiz_train1)
## 
## Deviance Residuals: 
##       Min         1Q     Median         3Q        Max  
## -0.065453  -0.003284  -0.003284  -0.003284   0.126086  
## 
## Coefficients:
##               Estimate Std. Error t value Pr(>|t|)    
## (Intercept)  1.074e+00  8.211e-04  1308.2   <2e-16 ***
## t           -8.706e-03  8.764e-06  -993.4   <2e-16 ***
## ---
## Signif. codes:  0 '***' 0.001 '**' 0.01 '*' 0.05 '.' 0.1 ' ' 1
## 
## (Dispersion parameter for gaussian family taken to be 0.001892997)
## 
##     Null deviance: 1882.836  on 7710  degrees of freedom
## Residual deviance:   14.593  on 7709  degrees of freedom
## AIC: -26458
## 
## Number of Fisher Scoring iterations: 2
\end{verbatim}

\begin{verbatim}
## 
## 载入程辑包:'psych'
\end{verbatim}

\begin{verbatim}
## The following objects are masked from 'package:ggplot2':
## 
##     %+%, alpha
\end{verbatim}

\begin{verbatim}
## 载入需要的程辑包:carData
\end{verbatim}

\begin{verbatim}
## 
## 载入程辑包:'car'
\end{verbatim}

\begin{verbatim}
## The following object is masked from 'package:psych':
## 
##     logit
\end{verbatim}

\begin{verbatim}
## The following object is masked from 'package:dplyr':
## 
##     recode
\end{verbatim}

\begin{verbatim}
## Call:corr.test(x = quiz_train1[, 4:5])
## Correlation matrix 
##         correct  t
## correct       1 -1
## t            -1  1
## Sample Size 
## [1] 7711
## Probability values (Entries above the diagonal are adjusted for multiple tests.) 
##         correct t
## correct       0 0
## t             0 0
## 
##  To see confidence intervals of the correlations, print with the short=FALSE option
\end{verbatim}

\begin{verbatim}
##         predict
## Observed    0    1
##        0 4444    0
##        1    0 3267
\end{verbatim}

\begin{verbatim}
## [1] 0
\end{verbatim}

\begin{verbatim}
## [1] 0
\end{verbatim}

\includegraphics{Report_files/figure-latex/unnamed-chunk-13-1.pdf}

\end{document}
